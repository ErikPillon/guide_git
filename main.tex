\documentclass{article}

\usepackage{listings}

\author{Erik Pillon}
\title{Git Guide: from zero to hero}
\date{\texttt{erik.pillon@uni.lu}}

\begin{document}
\begin{abstract}
	This is a list of tricks and nice commands that I collected during my experience with git.
\end{abstract}	

\section{Configuring a remote for a fork}
After forking a repository and branching the project, we would like to keep the \texttt{master} branch to be updated with the original repository. 
This can be easily achieved in two steps: 
\begin{itemize}
	\item You must configure a remote that points to the upstream repository in Git to sync changes you make in a fork with the original repository. This also allows you to sync changes made in the original repository with the fork.
	\item Sync a fork of a repository to keep it up-to-date with the upstream repository.
\end{itemize}

\subsection{Add an \texttt{upstream} repository}
For listing the current configured remote repository for your fork we have\\ %\vspace{0.1cm}

\texttt{git remote -v}\\ %\vspace{0.2cm}

and then specify a new remote upstream repository that will be synced with the fork:\\

\texttt{git remote add upstream https://github.com/\textit{ORIGINAL\_REPOSITORY}.git}\\

We can now verify the new upstream repository you've specified for your fork; if everything is successful, we'll have now 4 entries for the following command\\ %\vspace{0.1cm}

\texttt{git remote -v}
\subsection{Syncing a fork}

\end{document}